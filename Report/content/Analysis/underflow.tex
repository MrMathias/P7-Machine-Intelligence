\subsection{Underflow}
    Arithmetic underflow is a condition that can happen in computers, underflow is caused when a number is smaller than the minimum non-negative number a computer can store, and will therefore cause rounding errors bigger than usual. This can be seen by looking at how a computer stores numbers in binary, by looking at a 2 byte binary number with one byte before the decimal point and one byta after we get:\\

\begin{table}[!h]
    \begin{tabular}{|l|l|l|}
        \hline
        Bit number & Value    & Decimal \\ \hline
        3          & $2^3$    & 8       \\ 
        2          & $2^2$    & 4       \\ 
        1          & $2^1$    & 2       \\ 
        0          & $2^0$    & 1       \\ 
        -1         & $2^{-1}$ & 0.500   \\ 
        -2         & $2^{-2}$ & 0.250   \\ 
        -3         & $2^{-3}$ & 0.125   \\ 
        -4         & $2^{-4}$ & 0.0625  \\
        \hline
    \end{tabular}
\end{table}

    This makes the smalles number $0.0625$, any value less than this will get rounded either down to zero or up to $0.0625$. The smallest number in binary is:

    $$0000.0001 = 0.0625$$

\subsubsection{Does it matter}
asd

\subsubsection{How to work around it}
asd
