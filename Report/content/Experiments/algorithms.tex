\subsection{Algorithms}
In this section a number of algorithms are described, which all try to learn the parameters of a \gls{hmm} given a set of training sequences $D$.
In other words, each algorithm will try to find a model that makes the generation of the particular training sequences most likely.
Since a \gls{hmm} can be represented by the use of either matrices or a graph like structure, any of the algorithms may output either a matrix or graph representation of a \gls{hmm}, denoted by $M$ or $G$ respectively.
By $LL(M)$ or $LL(G)$ we denote the likelihood of the training sequences given the model.

Any algorithm introduced in the following may use the Baum Welch algorithm as a subroutine. 
To make a fair comparison between the algorithms described here and the Baum Welch algorithm, they all have two important input parameters, $n$ and $t$.
$n$ is the maximum allowed nodes or states to be used, and $t$ is the minimum threshold allowed for any use of the Baum Welch algorithm.
Changing the value of $n$ or $t$ could in turn improve an algorithm, and hence a fair comparison between algorithms requires fixed values for $n$ and $t$.
By $BW_t(M, D)$ we denote the \gls{hmm} obtained after running Baum Welch on the \gls{hmm} $M$ using the training sequences $D$, and iterating as long as each iteration increase the likelihood by at least $t$.
By $BW^i(M, D)$ we denote the \gls{hmm} obtained after running $i$ iterations of Baum Welch on the \gls{hmm} $M$ using the training sequences $D$.
If a \gls{hmm} is represented by a graph, we denote it $G$, and where the similar notations $BW_t(G, D)$ and $BW^i(G, D)$ are used.

\subsubsection{Greedy Extend}
This algorithm takes an input $(n, t, s, D)$.
Initially, a graph representation $G$ of a single state \gls{hmm} is created. The single node has initial probability 1, loops to itself with probability 1, and its emission probabilities for each of the $s$ symbols are chosen randomly and normalised.

Through a number of iterations $i = 1, ..., n$, do the following:
\begin{itemize}
\item Repeat 10 times:
	\begin{itemize}
	\item $G'$ = $(V(G) \cup \{y'\}, E(G))$, where $y'$ is a new node with a random initial probability in range $[0, 1]$ having random emission probabilities for all $s$ symbols, which sums to $1$.
	\item Randomly choose a set of nodes $Y = \{y_1, y_2, ... , y_l\}$ from $V(G')$, where $l = \lceil \log |V(G')| \rceil$ and $\forall a,b: y_a \neq y_b$.
	\item For each $y \in Y$, the transitions $(y, y')$ and $(y', y)$ are added to $E(G')$ with random transition probabilities.
	\item Normalise $G'$.
	\item If $LL(BM^{10}(G', D)) > LL(G)$, let $G =LL(BM^{10}(G', D))$.
	\end{itemize}
\end{itemize}

Return $BW_t(G, D)$.
\subsubsection{Sparse Baum Welch}
This algorithm takes an input $(n, t, s, D)$ and creates a \gls{hmm} $M$ with $n$ states and $s$ symbols.
All parameters are initialized randomly but with the constraint that each state has exactly $\lceil log(n) \rceil$ outgoing transitions.
Which transitions to discard are chosen randomly.
Return $BW(M_t, D)$.