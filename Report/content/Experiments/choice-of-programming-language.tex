\section{Choice of programming language}
Initially, Python was used for experimenting with the different algorithms (written in Python) which is provided at the Pautomac website.
Python is advantageous in the time it takes to create new or alter existing algorithms.
As it is a dynamically typed and a very concise language, one can quickly write a lot of code.
However, when continuously running benchmarks to measure the performance of new algorithms, Python was too slow.
We tried to increase the performance by using different interpreters of Python, which includes the standard Python interpreter, Anaconda and PyPy.
Different packages were also tried which claimed to provide fast calculations of floating points of high precision.
In the end, we found that switching to C\# increased the performance considerably.
Despite the increase in performance, C\#'s static types have been very convenient when more people have been working on the same code.
Unlike when using Python, the C\# code has adequately served as documentation.