\bibliographystyle{IEEEtrans}
\bibliography{content/Theory/PCFGs}


\section{Probabilistic Context-Free Grammars}


\subsection{Context-Free Grammars}
A context-free grammar is a 4-tuple $(N, \Sigma, R, S)$, where
\begin{enumerate}
\item N is a finite set of Non-terminals.   
\item $\Sigma$ is a finite set of terminals 
\item R is a finite set of rules, on the form $X \rightarrow Y_1,Y_2 ... Y_n$,
where $X \in N$ and $Y_i \in N \cup \Sigma$ for $i = 1 ... n$
\item $S \in N$ is the start symbol
\end{enumerate}
\cite[p.104]{sipser}
\cite[p.1]{collins}

\subsubsection{Left-most derivation}
A left-most derivation is a sequence of string $s_1, s_2 ... s_n$ such that
\begin{enumerate}
\item $s_1 = S$
\item $s_n \in \Sigma^*$ ($\Sigma^*$ is the set of all string over the alphabeth $\Sigma$)
\item each $s_i$ is derived from $s_{i-1}$ by picking the left-most Non-terminal in $s_{i-1}$ and replacing it with some $\beta$ where $X \rightarrow \beta$  is a rule in R.
\end{enumerate} 
\cite[p.2]{collins}

\subsubsection{Ambiguity}
A string $w$ is derived ambiguously in a context-free grammar G if it has two or
more different left-most derivations.
A Context-free grammar G is said to be ambiguous if it generates some string ambiguously.
\cite[p.108]{sipser}

\subsection{Chomsky Normal Form}
A Context-free grammar is in Chomsky normal form if every rule in R is of the form
\begin{enumerate}
\item $A \rightarrow BC$
\item $A \rightarrow \alpha$
\end{enumerate}

where $\alpha$ is any terminal symbol and A,B and C are any Non-terminals except that B and C may not be the start symbol S.
In addition the rule $S \rightarrow \epsilon$ is permitted.
\cite[p.109]{sipser}

\subsection{Probabilistic CFGs}










\subsubsection{Deriving a PCFG from a Corpus}













\subsubsection{Parsing with PCFG using the CKY Algorithm}

 % write about dynamic programming and how it relates to chomsky normal form.





