\section{Motivation}
For some systems, like many we find in nature, the underlying deterministic behaviour is not known. Instead only some observations about the system is known. Examples are speech recognition \cite{Rabiner89hmm}, pattern recognition for DNA or protein in bioinformatics \cite{Sakakibara2005} or creating intelligent actions in robotics \cite{Rivest1993}. For all of these problems the deterministic behaviour is hidden, where only a sequence of observations, or symbols, can be used to model the underlying systems behaviour.

When modelling such hidden systems, the common approach is to use a probabilistic model, with a probabilistic distribution over the observed data. The main question when modelling such an automata is then which model to use. More precisely, one want to find the model, which produces the largest probability to generate the observed symbol sequence. Two well known models are the PFA \cite{pazintroduction} and the HMM \cite{Rabiner89hmm}, however many different models have been proposed for different tasks. For instance the HMM seems to be the most widely applied model, where the PCFG model is assumed to be stronger for modelling bioinformatic systems, as the CFG better handle the structures of RNA and proteins \cite{Sakakibara2005}.