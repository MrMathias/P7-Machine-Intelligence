\section{PautomaC Competition}

\subsection{PautomaC}

PautomaC stands for Probabilistic Automata learning Competition. To
win, the competitors had to try and create the most accurate algorithm
for learning what are called \gls{npfa}. The aim of this competition was to improve
the existing knowledge on \gls{npfa} learning algorithms.


\subsubsection{Reminder about PAs}

More commonly referred to as simply \gls{pa} or
sometimes Rabin Automaton, like the name of Michael O. Rabin who introduced
this concept in 1963, \gls{pa}s generalise the concept of \gls{nfa}. \gls{nfa}s can transition to zero or more states,
without requiring input symbols for state transitions. However, \gls{pa}s
introduce transitioning probabilities in the definition. Therefore,
whereas \gls{nfa}s all have a Deterministic equivalent, the introduction
of the probability concept in \gls{pa}s prevent the existence of \gls{dfa} equivalents,
unless all probabilities are equal to 0 or 1.

The languages \gls{pa}s recognise are called stochastic languages - of which
the regular ones are a subset. More about stochastic languages can
be found {[}here{]}. The number of stochastic languages is uncountable.
By their definition, \gls{pa}s extend the Markov Chain concept. More formally:

\begin{description}
	\item [{P:}] probability of transitioning to a particular state
	\item [{Q:}] finite set of states
	\item [{$q_{0}$:}] probability vector of being in a particular initial
	state
	\item [{$\varSigma$:}] finite set of input symbols
	\item [{$\delta:Q\times\varSigma\longrightarrow P(Q)$:}] transition function
	\item [{F:}] finite set of final states, with $F\subset Q$
\end{description}

In addition, the transition matrix T from a state q to a state q'
while producing the symbol s must have the following property:

\[
\sum_{q'}[T_{s}]_{qq'}=1
\]


This means that, if a symbol s is produced, the probability of transitioning
from a state to another (or possibly the same) is 1. A Probabilistic
Automaton thus has:
\begin{itemize}
\item A probability of being in a particular initial state
\item A probability of producing a symbol depending on the current state
\item A probability of transitioning from a state to another, depending
on the symbol produced
\item A probability of ending the sequence when in a final state
\end{itemize}
Albeit complex, \gls{pa}s have many applications. Indeed, in most cases,
the only available data is the observed output. It is thus necessary
to learn how to make a model out of these observed sequences to understand
them. \gls{dna} sequences are a good example of this problem. The observed
symbols undoubtedly follow a model of some sort, yet unknown. The
goal of a \gls{pa} is to approximate this kind of model, and learning \gls{pa}s
ultimately leads to learning how \gls{dna} sequences are generated.


\subsubsection{Test generation}

All test files have been computer-generated. They include 4 model
types:
\begin{enumerate}
\item Markov Chains
\item Probabilistic Finite Automata (PFAs)
\item Deterministic Probabilistic Finite Automata (DPFAs)
\item Hidden Markov Models (HMMs)
\end{enumerate}
The manner in which those files have been generated is as follows:
\begin{itemize}
\item N for the number of states
\item A for the size of the alphabet (or number of observed symbols)
\item S for the scarcity of symbols
\item T for the scarcity of transitions\end{itemize}
\begin{description}
\item [{\textmd{The}}] ``S'' and ``T'' values can be confusing. Let's
clarify what they mean:

\begin{description}
\item [{S}] represents the average percentage of symbols produced by any
given state. In other words, each state will only produce S\% of the
whole alphabet A in average. Do note that two states may still produce
different symbols.
\item [{T}] represents the average percentage of all possible transitions
branching out from any given state. A transition here is a link from
one state to another, with a symbol associated to it. With this in
mind: for each state, out of all the possible transitions branching
out from it (each with their respective symbol and destination state),
only T\% will be generated.
\end{description}
\end{description}
The organisers entered those 4 values for each file. Finally, they
chose to set the number of initial states to T$\times$N and the number
of final states to S$\times$N. Everything else is entirely computer-generated. 


\subsubsection{The model files}

Generating a test model means that a model file is generated. This
file contains all the necessary information to generate sequences.
The information inside those files is displayed in a particular manner
described as follows:
\begin{description}
\item [{I:}] (state number)\\
Probability of starting the sequence in that state.\\
\textit{Note: the sum of all those probabilities is equal to 1.}
\item [{F:}] (state number)\\
Probably of ending the sequence when in that state\textit{.}\\
\textit{Note: those probabilities are independent of every other,
including the symbol and transition ones. Also, their sum is NOT equal
to 1.}
\item [{S:}] (state number, symbol number)\\
Probability of producing this symbol in this state.\\
\textit{Note: the sum of all the probabilities for any one state is
equal to 1.}
\item [{T:}] (starting state number, symbol number, destination state number)\\
Probability of reaching this destination state from this starting
state, given this symbol.\\
\textit{Note: the sum of all the probabilities sharing the same ``starting
state number'' and ``symbol number'' is equal to 1.}
\end{description}

\subsubsection{Rating}

At the time of the competition, the model and solution files were,
of course, not available. The competitors had to provide their solution
file to each test file. Those solution files did not contain a model,
though. What they contained instead was, for each sequence of the
test file, the probability for it to occur. The organisers then compared
their results to the actual probabilities inside their solution files,
and rated the competitors based on how close they were. The formula
they used is:

\[
2^{-(\sum_{x\epsilon TestSet}P_{r_{T}}(x)\times\log(P_{r_{C}}(x)))}
\]


where $P_{r_{T}}$ represents the organisers' probability for the
sequence x, and $P_{r_{C}}$ the candidates' one.

Worthy of note is that only the results mattered. This means that
the competitors\textquoteright{} models or numbers of states did not
matter, so long as they were able to approximate the behaviour of
the organisers\textquoteright{} models.


\subsubsection{Exploiting their data}

Our project also deals with finding a learning algorithm for these
automata. As such, the data available on PautomaC is a good starting
point. Additionally, now that the solution and model files have been
made publicly available, this gives us a means with which our algorithms
can be tested.